% Autogenerated from the org file---go edit that. 
 
\documentclass[11pt,final,hyperref={pdfpagelabels=false},xcolor=dvipsnames]{beamer} 
\mode <presentation> 
 
\usetheme{foiltex} 
 
\usepackage{epsfig,geometry,color,graphicx,hyperref,verbatim,bm,setspace,rotating,listings,xspace,amsmath,amsfonts} 
 
\begin{document} 
\def\nextslide#1{\end{frame} \begin{frame}{#1}} 
 
\def\liner{ 
\begin{center} 
\line(1,0){250} 
\end{center} 
} 
 
\lstset{columns=fullflexible, basicstyle=\small, 
emph={size\_t,apop\_data,apop\_model,gsl\_vector,gsl\_matrix,gsl\_rng,FILE},emphstyle=\bfseries} 
\def\setlistdefaults{\lstset{ showstringspaces=false,% 
 basicstyle=\small, breaklines=true,caption=,label=% 
,xleftmargin=.34cm,% 
,frameshape= 
,frameshape={nnnynnnnn}{nyn}{nnn}{nnnynnnnn} 
} 
\lstset{columns=fullflexible, basicstyle=\small, 
emph={size\_t,apop\_data,apop\_model,gsl\_vector,gsl\_matrix,gsl\_rng,FILE,math\_fn},emphstyle=\bfseries} 
} 
\setlistdefaults 
 
 
\title[]{Exchanges between Statistics and Machine Learning} 
\date{28/29 November 2017} 
\author[Klemens]{Ben Klemens} 
 
\def\Re{{\ensuremath{\mathbb R}}\xspace} 
 
\begin{frame}[plain] 
\maketitle 
\vfill {\small [The U.S. Treasury takes no position on the issues raised in this presentation.] }
\end{frame} 
\long\def\items#1{\begin{Witemize} 
    \setlength{\itemsep}{3pt} 
    \setlength{\parskip}{0pt} 
    \setlength{\parsep}{0pt} 
        \item #1 
                \end{Witemize} 
} 
\long\def\comment#1{\xspace} 
\setlength{\parskip}{0pt} 
\setlength{\parsep}{0pt} 
\savegeometry{beamer} 

\begin{frame}\frametitle{\color{red} Intro and Outline}
\items{ A broad discussion of what models have in common
 \item Examples of comparative statics, validation, and prediction across several types of model\pause
 \item What not to expect
\items{ Q: What is the right model to use?
\item A: That's not a well-formed question\pause
\item Q: What is the best way to evaluate a model?
\item A: That's not a well-formed question
}
}
\end{frame} \begin{frame}\frametitle{\color{red} Part I: What is a model?}
\end{frame} \begin{frame}\frametitle{\color{red} Problem statement}

\items{ "The word `model' in statistical literature usually refers to an equation to which one tries to fit data via regression analysis." [Complex Systems Modelling Group, {\em Modelling in Healthcare}, p 49]
}
\end{frame} \begin{frame}\frametitle{\color{red} Problem statement}
One afternoon, I tallied the models in the last 50 papers from the WB working paper
series and the U.S.\ Census Center for Economic Studies w.p. series.

\begin{center}
\begin{tabular}{lrr}
                                 &WB  & CES\\\hline
Papers with regressions only    &25   & 33\\
Papers including any other model     &7    &  4\\
Papers w/no model fitting       &18   & 13\\\hline
\end{tabular}
\end{center}

{\small WB: excluding no-model papers, 78\% regression} \\
{\small CES: excluding no-model papers, 89\% regression}\\

\end{frame} \begin{frame}\frametitle{\color{red} Probability vs Statistics}
\items{ Probability: Theorems. If the data has some property, as $N\to \infty$, something holds. (Law)
 \item Statistics: A summary of how the model designer sees the world. (Custom)
}

\end{frame} \begin{frame}\frametitle{\color{red} A statistical model links parameters and data via likelihoods}
\items{ estimation: data → parameters 
 \item RNG: parameters + arbitrary sequence → data 
 \item predict/conditional expected value: parameters + some data → other data 
 \item log likelihood, probability, entropy: parameters + data → a measure
}

\comment{

\end{frame} \begin{frame}\frametitle{\color{red} Apophenia plug}
\items{ Models could be combined: Bayesian updating → Bayesian networks
 \item Models could be cut: fix some parameters to known values
 \item Models could be mixed: $λ$(model 1) + (1-$λ$)(model 2)
 \item Models could be crossed: 
\items{ params = [model 1 params | model 2 params] 
\item data = [model 1 data | model 2 data] 
}
}
   \pause
\items{ Everybody wants to hide this grand uniformity
}
}

\end{frame} \begin{frame}\frametitle{\color{red} Remittances}
\comment{ sqlite3 remit.db "select sum(diff) d, namein from net\_remits group by namein order by d"|less }

\items{ Most sent out
\items{ United States:  263,225 million
\item Saudi Arabia:  87,412
\item United Arab Emirates:  62,242
\item Canada:  43,962
\item United Kingdom:  41,681
\item $\Sigma=1,263,791=174$ countries
}
 \item Most received in
\items{ China:  -116,573 million
\item India: -114,090
\item Philippines: -61,261
\item Mexico: -52,026
\item Pakistan: -38,761
\item $\Sigma=-1,263,791=197$ countries
}
}
[Run {\tt correlations()} in the demo script about here.]

\end{frame} \begin{frame}\frametitle{\color{red} Model I: guessing}
\items{ Belize \comment{net out: 158} \pause --- net out
 \item Ecuador \comment{net in: 3705} \pause --- net in
 \item Luxembourg \comment{net out: 717} \pause --- net out
 \item Malta \comment{net in: 291} \pause --- net in
 \item Iceland \comment{net in: 243} \pause --- net in
 \item Congo, both Republic and Dem. Republic \comment{net out: 126, 396} \pause --- net out
}

\end{frame} \begin{frame}\frametitle{\color{red} A list of models (1/2)}
\items{ generalized linear regression
\items{ may include nonlinear terms
\item logit, probit, et cetera
\item also includes systems of equations
\item an incomplete model--see below
}
 \item The Normal Distribution (params are $\mu, \sigma$)
\items{ Also, $\chi^2$, $t$, Zipf, Lognormal, Poisson \pause
}
 \item 'non parameteric models': a {\em lot} of parameters
\items{ A histogram is a model
\item Number of parameters may be a parameter
}
}
  

\end{frame} \begin{frame}\frametitle{\color{red} A list of models (2/2)}
\items{ Decision trees (parameters=cutpoints)
 \item Bayesian networks (parameters=cross of free submodel params)
\items{ Build a narrative piece by piece
}
 \item Support vector machines [categorization] (params=dividing line parms)
 \item neural networks (params=network activation params) \comment{ask audience here}
}
 
\end{frame} \begin{frame}\frametitle{\color{red} Understanding the parameters (comparative statics)}
\items{ ceteris paribus:
\items{ linear regression: $\beta$.
\item trees: find the relevant cutpoint; follow it
\item neural network: just try it \pause
}
 \item mutis mutandis
\items{ needs an underlying model for the data
\item linear regression: ¿¿¿???
}
}

[{\tt loop\_over\_models()}]

\end{frame} \begin{frame}\frametitle{\color{red} Part II: validation}

\end{frame} \begin{frame}\frametitle{\color{red} Parameter-based}
\items{ The parameters have some proven distribution $\rightarrow$ use that.
 \item Assumptions don't quite fit?
\items{ Find a theorem deriving the correct distribution. \pause
\item Or, just use the Normal distribution anyway.
}
}
     \comment{Here, ask for examples; shouldn't be a lot.}
\items{ Uses the model's likelihood function to evaluate the same model.
 \item Potentially difficult for non-parametric models past histograms.
}

\end{frame} \begin{frame}\frametitle{\color{red} Data-based}
\items{ How far does the model's implications about data diverge from the data?
 \item How accurate are its predictions?
 \item These are always available.
}

\end{frame} \begin{frame}\frametitle{\color{red} Replication}
\items{ The Bootstrap principle: draws from your sample $\approx$ draws from the population.
\items{ Given this, you can use it to estimate errors on the mean of nearly all parameters.
}
}
[{\tt loop\_over\_models(want\_boot=1)}]

\end{frame} \begin{frame}\frametitle{\color{red} An aside: entropy}
\items{ Has more real-world validity than most (law, not custom).
 \item Information loss in actual data $\rightarrow$ fake data from model
\items{ Kullback-Leibler divergence
\item Can be difficult: models truly falsified by the data have infinite divergence
}
 \item Adustment for unknown parameters $\rightarrow$ AIC.
\items{ Analogy: with unknown $\mu$, sample estimate of $\sigma \neq$ estimate with known $\mu$.
}
}

\end{frame} \begin{frame}\frametitle{\color{red} Train \& Test}
\items{ AKA Cross-validation
 \item The norm in ML, but usable for any model\pause
 \item We'll summarize via ROC (receiver operating characteristic)
}

[{\tt loop\_over\_models(want\_tt=1)}]

\end{frame} \begin{frame}\frametitle{\color{red} PS: What about Belize and Iceland?}

\begin{center}
\begin{tabular}{lrrrr}
&                Data    &Logit   &SVM     &Centroid\\
                United States & 1   &  1.00 $(1)$ &  1.00 $(1)$ &  0.67 $(1)$\\
                China         & 0   &  0.50 $(0)$ &  0.40 $(0)$ &  0.33 $(0)$\\
                Ecuador       & 0   &  0.29 $(0)$ &  0.25 $(0)$ &  0.33 $(0)$\\
                Malta         & 0   &  0.30 $(0)$ &  0.27 $(0)$ &  0.33 $(0)$\\
                Iceland       & 0   &  0.38 $(0)$ &  0.36 $(0)$ &  0.40 $(0)$\\
                Belize        & 1   &  0.38 $(1)$ &  0.41 $(1)$ &  0.44 $(1)$\\
                Luxembourg    & 1   &  0.47 $(1)$ &  0.45 $(1)$ &  0.48 $(1)$\\
                Congo, Rep.   & 1   &  0.50 $(1)$ &  0.52 $(1)$ &  0.54 $(1)$
\end{tabular}
\end{center}
Mark (1) for $>.405$ and (0) for $<.405$

[Output from {\tt make\_guesses()}]



\end{frame} \begin{frame}\frametitle{\color{red} Conclusion slide}
\items{ Almost everything you can do with a regression, you can do with any model
\items{ The one exception is parameter-based testing, for a large subset of models
\item Use the wealth of data-space tools
}
 \item Almost every tool commonly used with other models, you can use with a regression
}

\end{frame} \begin{frame}\frametitle{\color{red} Discuss further, ask hard questions, get the code}
\items{ {\tt ben.klemens@treasury.gov}
 \item {\tt ben@klemens.org}
 \item {\tt github.com/b-k/ml\_for\_econometricians}
}

\end{frame}
\end{document}
